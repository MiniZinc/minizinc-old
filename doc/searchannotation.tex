\documentclass[a4paper]{article}
\usepackage{alltt}
\usepackage[hmargin=2cm,vmargin=2.5cm]{geometry}
\usepackage{fancyvrb}
% Separate paragraphs with a blank line rather than indentating the
% first line.
\setlength{\parindent}{0pt}
\setlength{\parskip}{\baselineskip}

%\input{version.tex}

\newcommand{\minizinc}		{Mini\-Zinc}
\newcommand{\flatzinc}		{Flat\-Zinc}

% This is some LaTeX voodoo to make verbatim use a small font.
\makeatletter
\g@addto@macro\@verbatim\small
\makeatother

% FlatZinc uses the MiniZinc version number.
%
\title{Search Annotations in \flatzinc\\
\smallskip
%\Large{Version \mznversion}
\Large{\textbf{Third External Draft}}
}
\author{G12, National ICT Australia (NICTA), Victoria, Australia}
\date{}

%=============================================================================%

\newcommand{\fz}[1]{\texttt{#1}}
\newcommand{\fzsf}[1]{\ensuremath{\mathsf{#1}}}
\newcommand{\fzra}{\ensuremath{r_1}}
\newcommand{\fzrb}{\ensuremath{r_2}}
\newcommand{\fzxa}{\ensuremath{x_1}}
\newcommand{\fzxb}{\ensuremath{x_2}}
\newcommand{\fzxk}{\ensuremath{x_k}}
\newcommand{\fzxi}{\ensuremath{x_i}}
\newcommand{\fzya}{\ensuremath{y_1}}
\newcommand{\fzyb}{\ensuremath{y_2}}
\newcommand{\fzyk}{\ensuremath{y_k}}
\newcommand{\fzyi}{\ensuremath{y_i}}
\newcommand{\fzn}{\ensuremath{n}}
\newcommand{\fzv}{\ensuremath{v}}
\newcommand{\fzvj}{\ensuremath{v[j]}}
\newcommand{\fzi}{\ensuremath{i}}
\newcommand{\fzj}{\ensuremath{j}}
\newcommand{\fzk}{\ensuremath{k}}
\newcommand{\fzpredname}{\fzsf{predname}}
\newcommand{\fztype}{\fzsf{type}}
\newcommand{\fzparamtype}{\fzsf{paramtype}}
\newcommand{\fzvartype}{\fzsf{vartype}}
\newcommand{\fzargname}{\fzsf{argname}}
\newcommand{\fzvarname}{\fzsf{varname}}
\newcommand{\fzparamname}{\fzsf{paramname}}
\newcommand{\fzarg}{\fzsf{arg}}
\newcommand{\fzobjfn}{\fzsf{objfn}}
\newcommand{\fzvars}{\fzsf{vars}}
\newcommand{\fzliteral}{\fzsf{literal}}
\newcommand{\fzarrayliteral}{\fzsf{arrayliteral}}
\newcommand{\fzannotation}{\fzsf{annotation}}
\newcommand{\fzannotationname}{\fzsf{annotationname}}
\newcommand{\fzannotationarg}{\fzsf{annotationarg}}
\newcommand{\fzsearchannotation}{\fzsf{search\_annotation}}
\newcommand{\fzvarchoiceannotation}{\fzsf{variable\_selection}}
\newcommand{\fznassignmentannotation}{\fzsf{branching\_decision}}
% \newcommand{\fzstrategyannotation}{\fzsf{strategyannotation}}
\newcommand{\fzbnf}{\fzsf{::=}}
\newcommand{\fzalt}{\fzsf{|}}

\newcommand{\vblabel}[1]	{\large\textbf{#1}}
\newcommand{\vbcmd}		{>}
\newcommand{\vbout}		{\slshape}

\DefineVerbatimEnvironment{Zinc}{Verbatim}
    {samepage,frame=single,framesep=0.25em,xleftmargin=0.15em,xrightmargin=0.15em}

%=============================================================================%

\begin{document}

\maketitle

% \newpage

\tableofcontents

\newpage

\section{Introduction}

This document is the specification of the \flatzinc{} search annotations.
Search annotations are optional suggestions to the solver concerning how search
should proceed.  An implementation is free to ignore any annotations it does
not recognise, although it should print a warning on the standard error stream
if it does so.

\section{Basic Tree Search}
\label{sec:basic}
%While a \flatzinc{} implementation is free to ignore any or all annotations in
%a model, it is recommended that implementations at least recognize the
%following standard search annotations for solve goals.

A basic search annotation, \fzsearchannotation{}, can be one of the following:

\begin{tabular}{l}
\fz{
\{bool, int, set\}\_search(\fzvars, \fzvarchoiceannotation, \fznassignmentannotation)}
\end{tabular}

where \fzvars{} is an array variable name or an array literal specifying the
variables of the corresponding type to be assigned (ints, bools, or sets
respectively).

The search annotation dealing with float variables takes as additional argument the desired \emph{precision}:

\begin{tabular}{l}
\fz{
float\_search(\fzvars, prec, \fzvarchoiceannotation, \fznassignmentannotation)
}
\end{tabular}

The search on floats terminates as soon as the precision is within \texttt{prec}.

Note that all searches presented here must be \emph{complete} (i.e., exhaustive search). Hence, if the defined search only considers a subset of the variables and
fixing all of them to values does not yield a complete assignment (e.g.\ by
constraint propagation), \emph{default} search is invoked to fix all remaining
unassigned variables.  For a specification of default search please refer to
Section~\ref{defaultsearch}. Note that search applied on the empty set of variables 
succeeds. Section~\ref{sec:manipulators} introduces several manipulators to the searches introduced here. 

The search annotations \fz{\{int, set\}\_search\_all} do not take \fzvars{} as an
argument.  Instead it causes search to consider all integer/set variables contained
in a given problem instance, including auxiliary ones introduced during model
transformation.

\begin{tabular}{l}
\fz{
\{int, set\}\_search\_all(\fzvarchoiceannotation, \fznassignmentannotation)
}
\\
\end{tabular}


The arguments \fz{\fzvarchoiceannotation} and \fz{\fznassignmentannotation}
used in the search annotation are explained in Sections~\ref{sec:varselect}
and~\ref{sec:valueselect}. 

\subsection{Variable Selection}
\label{sec:varselect}
\fzvarchoiceannotation{} specifies how the next variable to be branched on is
chosen at each choice point.  Possible choices are as follows:

\begin{tabular}{p{.45\linewidth}p{.5\linewidth}}
\fz{\{reverse\_\}input\_order}
& Choose variables in the (reverse) order they appear in \fzvars.\\
\fz{random\_order}
& Choose variables at random (based on uniform distribution) from \fzvars.\\
\fz{\{min,max\}\_\{lb,ub\}}
& Choose the variable with the smallest/largest lower/upper bound.\\
\fz{\{min,max\}\_dom\_size}
& Choose the variable with the smallest/largest domain.\\
\fz{\{min,max\}\_degree}
& Choose the variable that appears in the smallest/largest number of
  constraints.\\
\fz{\{min,max\}\_\{lb,ub\}\_regret}
& Choose the variable with the smallest/largest difference between the two
smallest/largest values in its domain.\\
\fz{\{min,max\}\_dom\_size\_degree}
& Choose the variable with the smallest/largest ratio of domain size and degree.\\
\fz{\{min,max\}\_dom\_size\_weighted\_degree}
& Choose the variable with the smallest/largest ratio of domain size and
  weighted degree.\\
\fz{\{min,max\}\_impact}
& Choose the variable that caused the smallest/highest search space reduction
  in the past.\\
\fz{\{min,max\}\_activity}
& Choose the variable that has minimal/maximal activity score (e.g., based on VSIDS).\\
\end{tabular}

Note that implicitly tie breaking based on \texttt{input\_order} takes place when several variables have exactly the same score according to the used variable selection strategy. It is also possible to use the annotation \texttt{default}
as variable selection strategy which results in employing the default 
variable selection strategy  (which might differ depending on the type of the variable).

Multiple variable selection strategies can be combined using the \fz{\fzvarchoiceannotation} constructor \texttt{seq\_vss}. It selects a variable according to a sequence of strategies, whose relevance decreases from left to right. 

\begin{tabular}{l}
\fz{
seq\_vss([\fzvarchoiceannotation, \ldots, \fzvarchoiceannotation])
}
\end{tabular}

For instance, the following statement uses the variable selection strategy \texttt{min\_lb} with tie breaking based on the \texttt{reverse\_input\_order} of the variables:

\begin{tabular}{l}
\fz{
seq\_vss([min\_lb, reverse\_input\_order])
}
\end{tabular}

Note that while \texttt{input\_order} is implicitly used to perform tie breaking, it will never be applied in this example.
 
The following annotations on variable ordering strategies can be used to 
combine and manipulate variable scores. 
The first annotation can be used to \emph{weigh} the score of the employed  \fzvarchoiceannotation:

\begin{tabular}{l}
\fz{
weight\_score(\fzvarchoiceannotation, <Weight>)
}
\end{tabular}
where \texttt{<Weight>} denotes a float number.

The score of the used  \fz{\fzvarchoiceannotation} is multiplied by the given \texttt{<Weight>}. Note that the resulting score is rounded to the closest integer in order to promote further tie breaking.
The second annotation can be used to sum multiple individual scores and the resulting score is
then used to rank the variables accordingly:

\begin{tabular}{l}
\fz{
sum\_score([\fzvarchoiceannotation, \ldots, \fzvarchoiceannotation])
}
\end{tabular}

In order to use the weighted sum of the individual scores to rank the variables can then be formulated as follows:

\begin{tabular}{l}
\fz{
sum\_score([weight\_score(\fzvarchoiceannotation, <Weight>), \ldots, weight\_score(\fzvarchoiceannotation, <Weight>)])
}
\end{tabular}


\subsection{Branching Decisions}
\label{sec:valueselect}
\fznassignmentannotation{} specifies how the values of the selected variable
should be constrained.  Possible choices are as follows:

\begin{tabular}{p{.35\linewidth}p{.6\linewidth}}
\fz{\{assign,exclude\}\_\{lb,ub\}}
& Assign/exclude the smallest/largest value in the variable's domain:\\
& $x = \min\{d(x)\}$ $\swarrow\searrow$ $x \neq \min\{d(x)\}$ or\\
& $x = \max\{d(x)\}$ $\swarrow\searrow$ $x \neq \max\{d(x)\}$ or\\
& $x \neq \min\{d(x)\}$ $\swarrow\searrow$ $x = \min\{d(x)\}$ or\\
& $x \neq \max\{d(x)\}$ $\swarrow\searrow$ $x = \max\{d(x)\}$, resp.\\

\fz{\{assign,exclude\}\_mean}
& Assign/exclude the mean value in a variable's domain.\\
& $x=m$  $\swarrow\searrow$ $x \neq m$ or \\
& $x\neq m$  $\swarrow\searrow$ $x=m$, resp., 
where $m=\left\lfloor\frac{\min\{d(x)\} + \max\{d(x)\}}{2}\right\rfloor$ \\


\fz{\{assign,exclude\}\_median}
& Assign/exclude the median value in a variable's domain.\\
& $x=m$  $\swarrow\searrow$ $x \neq m$ or \\
& $x\neq m$  $\swarrow\searrow$ $x=m$, resp., 
where $i=\left\lfloor\frac{|d(x)| + 1}{2}\right\rfloor$, \mbox{$d(x)=\{a_1,...,a_n\}$} with $a_1 < a_2 < \ldots a_n$, and $m=a_i$.\\

\fz{\{assign,exclude\}\_random}
& Assign/exclude a value uniformly\footnote{Other distributions like Gaussian could be considered as well.} sampled from the variable's domain.\\
& $x=r$ $\swarrow\searrow$ $x \neq r$ or\\
& $x \neq r$ $\swarrow\searrow$ $x = r$, resp., where $r=random\{d(x)\}$\\

\fz{\{assign,exclude\}\_impact\_\{min,max\}}
& Assign/exclude the value in the variable's domain with the minimal/maximal impact score.\\

\fz{\{assign,exclude\}\_activity\_\{min,max\}}
& Assign/exclude the value in the variable's domain with the minimal/maximal activity score (e.g., based on VSIDS).\\

\fz{\{include,exclude\}\_\{min,max\}}
& Include/exclude the smallest/largest uncertain domain value of a set variable:\\
& $x = d_s(x) \cup \{d\}$ $\swarrow\searrow$ $x= d_s(x)-\{d\}$  or\\  
& $x = d_s(x) -\{d\}$ $\swarrow\searrow$ $x = d_s(x) \cup \{d\}$, resp., where \\
& $d=\min\{\max\{d_s(x)\} \setminus \min\{d_s(x)\} \}$ or \\
& $d=\max\{\max\{d_s(x)\} \setminus \min\{d_s(x)\} \}$, resp.\\

\fz{enumerate\_\{lb,ub\}}
& Enumerate values from the smallest to the largest/from the largest to
  the smallest in the variable's domain.\\
& $x=a_1\downarrow x=a_2 \downarrow \ldots \downarrow x=a_n$ or\\
& $x=a_n\downarrow x=a_{n-1} \downarrow \ldots \downarrow x=a_1$, resp.,
where \mbox{$d(x)=\{a_1,...,a_n\}$} with $a_1 < a_2 < \ldots a_n$ \\

\fz{bisect\_\{low,high\}}
& Bisect the variable's domain, excluding the upper/lower half first.\\
& $x \leq a $ $\swarrow\searrow$ $ x>a$ or\\
& $x \geq  a $ $\swarrow\searrow$ $ x<a$, resp.,
where $a=\left\lfloor\frac{\min\{d(x)\} + \max\{d(x)\}}{2}\right\rfloor$, resp.\\

\fz{bisect\_median\_\{low,high\}}
& Bisect the variable's domain based on the median, excluding the upper/lower half first.\\
& $x \leq m $ $\swarrow\searrow$ $ x>m$ or\\
& $x \geq m $ $\swarrow\searrow$ $ x<m$, resp.,
where $i=\left\lfloor\frac{|d(x)| + 1}{2}\right\rfloor$, \mbox{$d(x)=\{a_1,...,a_n\}$} with $a_1 < a_2 < \ldots a_n$ and $m=a_i$.\\

\fz{bisect\_random\_\{low,high\}}
& Bisect the variable's domain by splitting on a value selected at random excluding upper/lower half first.\\
& $x \leq r $ $\swarrow\searrow$ $ x>r$ or\\
& $x \geq r $ $\swarrow\searrow$ $ x<r$, resp., where $r=random\{d(x)\}$\\

\fz{bisect\_interval\_\{low,high\}}
& If the variable's domain consists of several intervals,
split the domain into the interval containing the smallest/largest values
and the remaining intervals.
Otherwise perform \fz{bisect\_low}/\fz{bisect\_high} on the variable's domain.\\
& $x \in [a,b]$ $\swarrow\searrow$ $x \not\in [a,b]$ or\\
& $x \not\in [a,b]$ $\swarrow\searrow$ $x \in [a,b]$, resp.,
  where $[a,b] \subset d(x)$ and $\exists.d\in[a,b]$ s.t. $d=min(d(x))$ or  $d=max(d(x))$ resp.\\

\fz{bisect\_impact\_\{min,max\}}
& Bisect the variable's domain, excluding the half with minimal/maximal impact score first.\\

\fz{bisect\_activity\_\{min,max\}}
& Bisect the variable's domain, excluding the half with minimal/maximal activity score first.\\
&
\end{tabular}

Of course, some branching decision specifications may be the same strategy;
e.g., \fz{bisect\_low} and \fz{enumerate\_lb} in \fz{bool\_search}. In addition,
note that \fz{bisect\_\{low,high\}} corresponds to \fz{bisect\_mean\_\{low,high\}}.

A branching decision is applied just once, and the variable may thus not yet be
fixed as a result (e.g.,  with domain bisection).  This variable may or may not
be selected immediately afterwards, depending on the specified variable
selection. It is also possible to use the annotation \texttt{default}
as domain splitting strategy which results in employing the default 
domain splitting strategy (which might differ depending on the type of the variable).

In addition, the user can specify a single strategy of the available domain
splitting strategies which then will be applied to all variables branched on.

It is also possible to provide multiple domain splitting strategies using the
annotation \texttt{dss} (which is of type \fz{\fznassignmentannotation)}).
When specifying this annotation a domain splitting strategy is picked at
\emph{random} per variable from the specified set of strategies\footnote{One
could also think of correlating the type of variable (e.g., based on domain
size) with particular domain splitting strategies.}.

\begin{tabular}{l}
\fz{dss([\fznassignmentannotation, \ldots,\fznassignmentannotation])}
\end{tabular}


%=============================================================================%


\section{Complex Search Constructs}
\label{sec:manipulators}

In this section we introduce several search templates that can be used to
construct more complex searches (e.g., support of interaction between
searches).  Note that all search templates presented here are of type
\fzsearchannotation{} and are therefore applicable whenever the standard
\fzsearchannotation{} is. 
Please note that some search manipulators introduced here cause search to be \emph{incomplete}.

\subsection{Sequential Search}

\emph{Sequential Search} defines multiple searches on variable subsets
$X_1, X_2, \ldots, X_n$ of the decision variables $X$ in the given problem
instance in a particular order.  Typically, $X_1 \cup X_2 ...  \cup X_n = X$.
The searches are executed in the specified order.  If a search
fails, sequential search backtracks to the previous search in the sequence.

\begin{Zinc}[label={\vblabel{Sequential Search}}]
seq_search([<SearchAnn(X1)>, <SearchAnn(X2)>, ... <SearchAnn(Xn)>])
\end{Zinc}

As an example consider the following sequential search annotation:

\begin{Zinc}[label={\vblabel{Example: Sequential Search}}]
seq_search([search(x1, min_dom_size_weighted_degree, enumerate_lb),
            search(x2, min_dom_size, bisect_low)
           ])
\end{Zinc}

First search takes places on the variables in $x1$ and once all variables
contained in $x1$ have been assigned, the search on $x2$ is invoked.  If the
search on $x2$ fails, then the sequential search annotation backtracks to the
previous search on $x1$ in order to determine a new setting of the variables in
$x1$.  If there are no more feasible assignments left among the variables in
$x1$, sequential search terminates without a solution.


\subsection{Parallel Search}

If possible the searches specified inside of the \emph{Parallel Search}
annotation are invoked in parallel and this search construct terminates as soon
as one of the specified searches is able to determine an answer to the given
problem%
\footnote{In the current implementation the searches are executed sequentially
in the order as they are specified.  Hence, the first successful search in the
sequence causes parallel search to terminate.}.
Note that all searches inside the \emph{Parallel Search} annotation are
independent of each other and no backtracking between searches as in sequential
search takes place.

\begin{Zinc}[label={\vblabel{Parallel Search}}]
par_search([<SearchAnn>, <SearchAnn>, ..., <SearchAnn>])
\end{Zinc}


\subsection{Limited Search}

\emph{Limited Search} bounds the search process by measures like time (in
seconds), number of failures or number of search nodes.  Limits may be
nested; a search is always constrained by the tightest limit it is
scoped by.  The \emph{limit} construct is also of type \fzsearchannotation{}.

\begin{Zinc}[label={\vblabel{Limited Search}}]
limit_search(<Measure>, <Limit>, <SearchAnn>)
\end{Zinc}

\emph{Measure} can take one of the following values: 

\begin{tabular}{p{.10\linewidth}p{.6\linewidth}}
\fz{fails} 
& Number of failures/conflicts\\
\fz{nodes} 
& Number of search nodes traveresed\\
\fz{time}
& Time in seconds
\end{tabular}


\subsection{Restarting Search}

Any search scoped by a restart annotation backtracks to the root every time the
restart condition is hit.  As with \emph{Limited Search} restarts can be
defined on several measures (e.g., number of failures).  Note that in contrast
to limits, restarts cannot occur nested.  In general, the restart annotation
has the following format:

\begin{Zinc}[label={\vblabel{Restarting Search}}]
restart_<Type>(<Type-Specific-Parameters>, <SearchAnn>)
\end{Zinc}

The following types of restart strategies are supported:
\emph{Geometric} and \emph{Luby} based restarts:

\begin{Zinc}[label={\vblabel{Geometrically Restarting Search}}]
restart_geometric(<Increment>, <InitialLimit>, <Measure>, <SearchAnn>)
\end{Zinc}

\begin{Zinc}[label={\vblabel{Luby Restarting Search}}]
restart_luby(<InitialValue>, <MaximalValue>, <Measure>, <SearchAnn>)
\end{Zinc}


\subsection{Sample Search}

The \emph{Sample Search} construct allows information to be passed to
variable ordering heuristics\footnote{One could also consider sampling for
value ordering heuristics or for both ordering strategies.} from one search to
a subsequent search.  The aim of this template is to enable the collection of
information relevant to a variable ordering heuristic in order to support
``warm-starts''.  The sample search template takes a \fzvarchoiceannotation,
and the corresponding information is collected during the first limited search.
Note that variable selection heuristics that depend on static information only
are not a sensible choice (e.g., \texttt{input\_order}).

\begin{Zinc}[label={\vblabel{Sample Search}}]
sample_search([<VarChoiceAnn>, limit_search(<Measure>, <Limit>, <SearchAnn>), <SearchAnn>])
\end{Zinc}

The following is an example for an application of the sample search template:

\begin{Zinc}[label={\vblabel{Example: Sample Search}}]
sample_search(impact,
    limit_search(time, 5,
        restart_geometric(1.15, 200.0, nodes,
            int_search(x, min_lb, dss([bisect_low, bisect_high]))))
    int_search(x, seq_vss([impact, min_dom_size]), enumerate_lb)
)
\end{Zinc}

This sample search first performs a search limited by $5$ seconds that makes
use of restarts and branches on variables in $x$.  During this first search
information relevant to the \texttt{impact} heuristic is gathered (e.g., search
space reduction per variable).  The second search accesses this information to
perform its initial branching decisions.  Note, however that this information
is continued to be updated as in the regular \texttt{impact} heuristic.


\subsection{Backdoor Search}

The backdoor search annotation tries to determine a subset of variables that
simplify the problem in an essential fashion once they are fixed to a value.
In order to determine such variables, a limited search is performed to collect
information on the variables in the given problem instance.  As a result, the
variables are partitioned into \emph{backdoor} variables and \emph{remaining}
variables.  Then a sequential search is executed --- first on \emph{backdoor}
variables and second on \emph{remaining} variables.

In the following specification of \emph{Backdoor Search}, $X$ denotes an
arbitrary subset of the variables contained in the problem instance, and
'\fz{SearchAnn(X)}' denotes a normal search construct.  The
annotations '\fz{backdoor\_vars}' and '\fz{remaining\_vars}' are replaced by
\emph{backdoor}/\emph{remaining} variables respectively based on the
information collected during '\fz{SearchAnn(X)}'.  The variables in $X$ are sorted
by the given \fzvarchoiceannotation{} variable score after the first search has
terminated.  The highest ranked variables that fall within the given
\emph{ratio} according to the used variable ordering heuristic correspond to
the set of \emph{backdoor} variables $B$ branched on in the first search in the
sequential search sequence, while the \emph{remaining} variables $R$ are the
ones branched on in second search in the sequential search sequence
(with $X = B \cup R$).

\begin{Zinc}[label={\vblabel{Backdoor Search}}]
backdoor_search(<VarChoiceAnn>, <Ratio>,
    limit_search(<Measure>, <Limit>, <SearchAnn(X)>),
        seq_search([
            <SearchAnn(backdoor_vars)>,
            <SearchAnn(remaining_vars)>,
        ])
)
\end{Zinc}

\subsection{Limited Discrepancy Search}
This manipulator results in restricting the scoped search to the specified 
discrepancy.

\begin{tabular}{l}
\fz{
lds(<Discrepancy>, <SearchAnn>)
}
\end{tabular}

In the following we show an example usages of the \fz{lds} search manipulator.
This example limits the search scoped by the \fz{lds} manipulator to only 
allow search with discrepancy $1$:
\begin{Zinc}[label={\vblabel{Example: Limited Discrepancy Search}}]
% Restrict scoped search on X to discrepancy 1
lds(1, int_search(X, min_degree,  bisect_low)  
\end{Zinc}

\subsection{Assign Once}
This manipulator results in assigning all free variables to a value according
to the given search annotation. An assignment is only made once, and backtracking
does not take place.

\begin{tabular}{l}
\fz{
once<SearchAnn>)
}
\end{tabular}

In the following we show example usages of the \fz{once} search manipulator.
The first example shows how do define a standard search on the variables $X$, and then how to fix the variables $Y$ (and not to perform backtracking on variables in $Y$) using \fz{once}:

\begin{Zinc}[label={\vblabel{Example 1: Search Once}}]
seq_search([
    % Search that solves problem by branching on X
    int_search(X, min_degree,  bisect_low),  
    % Assign value (here: lower bound) to all remaining variables Y 
    once(int_search(Y, min_dom_size, min_lb)) 
]).
\end{Zinc}
The second  example shows how do define a standard search on the variables $X$, then how to fix the variables $Y$ (and not to perform backtracking on variables in $Y$) using \emph{once}, and then to search on variables $Z$. When search on $Z$ fails, \fz{seq\_search} backtracks directly over all variables in $Y$ to the search on X:

\begin{Zinc}[label={\vblabel{Example 2: Search Once}}]
seq_search([
    % Search that solves part of the problem involving X
    int_search(X, min_degree,  bisect_low)
    % Assign value (here: lower bound) to all remaining variables Y
    once(Y, min_dom_size, min_lb),
    % Search that solves part of the problem involving Z
    int_search(Z, min_degree,  bisect_low)
]).
\end{Zinc}


%=============================================================================%


\section{Default Search}
\label{defaultsearch}

Default Search is \flatzinc{} implementation dependent.
It is typically subject to change over time (e.g., by determining more
robust search strategies).
Default search is used when one of the following cases occurs:
\begin{itemize}
\item no search annotation specified in the instance;
\item the given search annotation is ignored
      (e.g., because it is not supported);
\item after applying the given search annotation unfixed variables remain.
\end{itemize}

The search specification used for default search in G12 is currently the
following:

\begin{Zinc}[label={\vblabel{Default Search}}]
seq_search([
    int_search(x_i, min_dom_size, bisect_low),
    bool_search(x_b, input_order, enumerate_lb),
    set_search(x_s, min_dom_size, enumerate_lb)
])
\end{Zinc}
where \fz{x\_i, x\_b, x\_s} denote all currently unassigned variables
of the respective type in the given problem instance.


%=============================================================================%


\section{Examples}

\begin{Zinc}[label={\vblabel{Search on X with Variable/Domain Ordering}}]
int_search(x, min_dom_size_weighted_degree, enumerate_lb)
\end{Zinc}

\begin{Zinc}[label=%
{\vblabel{Search on All Variables with Multiple Variable/Domain Orderings}}]
int_search_all(seq_vss([max_degree, min_lb]), dss([bisect_low, bisect_high]))
\end{Zinc}

\begin{Zinc}[label={\vblabel{Geometrically Restarting Search}}]
restart_geometric(100, 1.05, fails,
    int_search(x, min_dom_size_weighted_degree, enumerate_lb)
)
\end{Zinc}

\begin{Zinc}[label={\vblabel{Search with Nested Limits}}]
limit(time, 100,
    parallel_search([limit(fails, 1000,
        int_search(x, min_dom_size, bisect_low)),
        restart_geometric(100, 1.05, fails,
            int_search(y, impact, enumerate_lb)
        )
    ])
)
\end{Zinc}

\begin{Zinc}[label={\vblabel{Backdoor Search Example}}]
backdoor_search(dom_min_weighted_degree, 0.1,
    limit_search(time, 10,
        restart_geometric(1.15, 200.0, nodes,
            search(x, min_dom_size_weighted_degree, bisect_low)
        )
    ),
    seq_search([
        search(backdoor_vars, min_dom_size, bisect_low)
        search(remaining_vars, input_order, enumerate_lb)
    ])
)
\end{Zinc}


\end{document}

